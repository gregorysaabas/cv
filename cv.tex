\documentclass{res}
\usepackage{helvetica} % uses helvetica postscript font (download helvetica.sty)
\newsectionwidth{0pt}  % So the text is not indented under section headings
\usepackage{fancyhdr}  % use this package to get a 2 line header
\usepackage[utf8]{inputenc}
\renewcommand{\headrulewidth}{0pt} % suppress line drawn by default by fancyhdr
\newcommand{\inFrench}[1]{}
\newcommand{\inEnglish}[1]{#1}
\setlength{\headheight}{24pt} % allow room for 2-line header
\setlength{\headsep}{16pt}  % space between header and text
\setlength{\headheight}{24pt} % allow room for 2-line header
\pagestyle{fancy}
\rhead{ {\it Gregory Saabas}\\{\it p. \thepage} }
\cfoot{}
\topmargin=-0.7in
\addtolength{\textheight}{0.7in}

\begin{document}
\thispagestyle{empty} % this page has no header

{\bf\huge Gregory Saabas}
\inFrench{
	\hfill \textbf {github.com/gregorysaabas} \\
	Diplômé en science informatique de l'Université de Waterloo.
	\hfill \textbf {gregory.saabas@gmail.com} \\
	\hfill \textbf {Tél.}: 514-209-1931 \\
}
\inEnglish{
	\hfill \textbf \\ %{github.com/gregorysaabas} \\
	Computer Science graduate from the University of Waterloo.
	\hfill \textbf {gregory.saabas@gmail.com} \\
	\hfill \textbf {tel}: 514-209-1931 \\
}
\vspace{-15pt}

\begin{resume}
% --------------------------------
\section{Education}
\vspace{4pt}

\inFrench{
	% Bacc
	Université de Waterloo, Waterloo ON
	\hfill 2007-2012 \\
	{\sl Baccalauréat en science informatiquel}
}
\inEnglish{
	% Bacc
	University of Waterloo, Waterloo ON
	\hfill 2007-2012 \\
	{\sl Bachelor in Computer Science}
}


% --------------------------------
\inFrench{
	\section{Expériences professionnelles}
}
\inEnglish{
	\section{Professional experiences}
}
\vspace{6pt}

%\inFrench{
%	% Nuance
%	Nuance, Montréal\footnotemark
%	\hfill octobre 2017 jusqu'à présent \\
%	{\sl Développeur Java} \hfill (Stage universitaire 3)
%	\vspace{0.05in}
%
%	\begin{itemize} \itemsep -2pt
%		\item Ajout de fonctionnalités dans une API d'IPTV (Java);
%		\item Création et amélioration d'utilitaires pour tests (Bash, JavaScript et Python);
%		\item Développement d'une application de contrôle d'accès à base de rôles (RBAC) en Python;
%		\item Fonctionnalités d'exportation et d'importation de données d'un serveur LDAP à des serveurs Linux.
%	\end{itemize}
%
%	% Schneider-Electric
%	Schneider-Electric, Dollard-des-Ormeaux
%	\hfill janvier 2014 à août 2017 \\
%	{\sl Développeur} \hfill (Stage universitaire 2)
%	\vspace{0.05in}
%
%	\begin{itemize} \itemsep -2pt
%		\item Développement d'une solution Python de "Monitoring as a Service" pour Openstack;
%		\item Maintien et développement d'une application AngularJs avec des vues configurables;
%		\item Création de paquets RPM (RedHat) pour l'installation de la solution;
%		\item Participation à un processus agile.
%	\end{itemize}
%
%	% Invensys
%	Invensys, Dollard-des-Ormeaux
%	\hfill janvier à decembrel 2013 \\
%	{\sl Développeur}
%	\vspace{0.05in}
%
%	\begin{itemize} \itemsep -2pt
%		\item Conversion d’un site PHP vers une application web moderne en JavaScript (NodeJs et AngularJs);
%		\item Implémentation de fonctionnalités d’un système conventionnel de gestion d’utilisateurs;
%		\item Gestion de l’ensemble du projet de manière agile avec des tâches et un "board".
%	\end{itemize}
%
%	% Lyon
%	% Université Catholique de Lyon, Lyon, France
%	% \hfill avril à mai 2013 \\
%	% {\sl Développeur au service informatique} \hfill (Stage collégial)
%	% \vspace{0.05in}
%
%	% \begin{itemize} \itemsep -2pt
%	%	\item Création d’une application avec l’aide d’un ETL qui extrait les données d’une base de données, les modifie et les achemine à la base de données du système d’éducation public français;
%	%	\item Analyse et remplacement d’un logiciel désuet dans le but de créer une application plus conforme et extensible.
%	% \end{itemize}
%	
%	%\footnotemark[\value{footnote}]
%	%\footnotetext{Environnement de travail bilingue}
%}

\inEnglish{
      % Ubisoft R6 Magnum
	Ubisoft, Montréal
	\hfill August 2022 to April 2023 \\
	{\sl Online Developer}
	\vspace{0.05in}

	\begin{itemize} \itemsep -2pt
		\item Maintained and developed micro-services (ASP.NET Core) with REST APIs
		\item Added persistence for micro-services (Redis, MongoDB)
		\item Maintained project Devops (Gitlab-CI, Gitlab-CD, Helm, Kubernetes, Kibana, Grafana)
		\item Developed features following best practices (Gitlab Code Review, XUnit Tests, E2E Tests).
		\item Identified and created proposals for several infrastructural improvements.
	\end{itemize}

	% Nuance Communications DWS
	Microsoft Nuance, Montréal
	\hfill October 2019 to July 2022 \\
	{\sl Java Software Developer}
	\vspace{0.05in}

	\begin{itemize} \itemsep -2pt
		\item Implemented new endpoints for Spring micro-services communicating through GRPC
		\item Added persistence for micro-services (PostgreSQL)
		\item Setup and maintained project Devops (Gitlab-CI, Harness-CD, Helm, Kubernetes, Prometheus, Grafana)
		\item Debugged production using Azure Logs, metrics and tracing (Jaeger)
	    \item Developed features following best practices (Gitlab Code Review, JUnit Unit Tests, JUnit E2E Tests).
	\end{itemize}

      % Nuance Communications before Cerence
	Nuance Communications, Montréal
	\hfill October 2017 to September 2019 \\
	{\sl Application Software Developer}
	\vspace{0.05in}

	\begin{itemize} \itemsep -2pt
		\item Developed Dialog Applications using proprietary software and Javascript in an Agile work environment
		\item Filed several feature requests to the engine team to improve the Dialog Application development process
		\item Developed a tool to run integration tests on the Dialog Applications
	\end{itemize}

	% Schneider-Electric Redmine
	Schneider-Electric, Dollard-des-Ormeaux
	\hfill January 2015 to August 2017 \\
	{\sl Software Developer}
	\vspace{0.05in}

	\begin{itemize} \itemsep -2pt
		\item Set up and maintained an open source Issue Tracking web application (Redmine for Ruby on Rails) for use by nuclear engineers
		\item Deployed and maintained an nginx server and MSSQL server
		\item Performed data migrations across different versions of Redmine
		\item Developed several plugins to automate certain tasks within the application and improve usability
	\end{itemize}

	% Schneider-Electric HMI
	Schneider-Electric, Dollard-des-Ormeaux
	\hfill January to December 2014 \\
	{\sl HMI Application Developer}
	\vspace{0.05in}

	\begin{itemize} \itemsep -2pt
		\item Created and tested an HMI application to be used by nuclear power plant operators.
		\item Diagnosed and designed solutions for system design issues
	\end{itemize}

	% Invensys
	Invensys, Dollard-des-Ormeaux
	\hfill January to December 2013 \\
	{\sl Software developer}
	\vspace{0.05in}

	\begin{itemize} \itemsep -2pt
		\item Designed and developed internal tools to assist with integration of our Distributed Control System solution with clients preexisting solutions
		\item Created Windows Forms applications in C\#
		\item Produced Documentation for procedures and internal tooling
	\end{itemize}
}

% --------------------------------
\inFrench{
	\section{Connaissances informatiques}
}
\inEnglish{
	\section{Software expertise}
}
\vspace{6pt}

%\inFrench{
%	Langages de programmation
%	\vspace{0.05in}
%	\begin{itemize}
%		\item Java (Spring), Python, C\#, JavaScript, Rust, Bash.
%	\end{itemize}
%
%	Logiciels maîtrisés
%	\vspace{0.05in}
%	\begin{itemize}
%		\item GNU/Linux, Git, Eclipse,  IntelliJ Idea, Docker, Kubernetes, Vim.
%	\end{itemize}
%}

\inEnglish{
	Programming languages
	\vspace{0.05in}
	\begin{itemize}
		\item Java (Spring), C\#, Python, Go, JavaScript, Rust, Bash.
	\end{itemize}

	Software
	\vspace{0.05in}
	\begin{itemize}\itemsep -2pt
		\item GNU/Linux
		\item Git
		\item IntelliJ Idea
		\item Docker
		\item Kubernetes
		\item Gitlab-CI
	\end{itemize}
}

%\newpage

% --------------------------------
% Beginning of page 2
% --------------------------------
\inFrench{
	\section{Intérêts personnels}
}
\inEnglish{
	\section{Personal interests}
}
\vspace{6pt}

%\inFrench{
%	\vspace{6pt}
%	\begin{itemize} \itemsep -2pt
%		\item Logiciels libres et open source;
%		\item Développement Linux Kernel et projet Debian;
%		\item Algorithmie et intelligence artificielle;
%		% \item Nouvelles informatiques et avancées technologiques;
%		% \item Utilisation des technologies dans un contexte social;
%		\item Conception de jeux vidéos et ludification (gamification).
%	\end{itemize}
%}

\inEnglish{
	\vspace{6pt}
	\begin{itemize} \itemsep -2pt
		\item Free software and open source.
		\item Automation and Productivity tools and technology
		\item Code style: efficiency, maintainability, and composability
		\item Fighting games, board games
		\item Input devices (custom controllers, custom keyboards, steno keyboards)
	\end{itemize}
}

\inEnglish{
	\section{Languages}
}
\vspace{6pt}

\inEnglish{
	\vspace{6pt}
	\begin{itemize} \itemsep -2pt
		\item English: Native
		\item French: Native
		\item Japanese: Beginner
	\end{itemize}
}

%% --------------------------------
%\inFrench{
%	\section{Implication scolaire et accomplissements}
%}
%\inEnglish{
%	\section{Scholar involvement and accomplishments}
%}
%\vspace{6pt}
%
%%\inFrench{
%%	SORT: Sustainability through Object Recognition and Training
%%	\hfill January to April 2017 \\
%%	{\sl Technical leader} \hfill (Projet de fin d'études universitaires)
%%	\vspace{0.05in}
%%	\begin{itemize} \itemsep -2pt
%%		\item Création d'une application reconnaissant les déchets dans une image pour aider la communauté.
%%		\item Recherche de logiciel de reconnaissance d'images (OpenCV, Yolo).
%%		\item Déploiment d'une architecture à multiple application avec Docker et Docker-compose.
%%		\item Gestion d'un projet open source sur GitHub (release, pull request, board, etc.).
%%	\end{itemize}
%%
%%	{\sl Membre du club Formule ÉTS} \hfill (Club étudiant)
%%	\vspace{0.05in}
%%	\begin{itemize} \itemsep -2pt
%%		\item Création d’un document de vision ainsi qu’un document de spécification des requis pour une application de télémétrie en temps réel;
%%		\item Prototypage d'une application de télémétrie en temps réel en JavaScript.
%%	\end{itemize}
%%
%%	{\sl Membre du club Conjure} \hfill (Club étudiant)
%%	\vspace{0.05in}
%%	\begin{itemize} \itemsep -2pt
%%		\item Création d’un jeu en C\# avec le logiciel Unity;
%%		\item Gestion du projet.
%%	\end{itemize}
%%
%%	% {\sl Promotion du programme collégial} \hfill (Cégep)
%%	% \vspace{0.05in}
%%	% \begin{itemize} \itemsep -2pt
%%	% 	\item Entrevue avec le journal régional pour la promotion du programme collégial;
%%	% 	\item Participation aux portes ouvertes pour la présentation des projets accomplis durant mon parcours.
%%	% \end{itemize}
%%
%%	{\sl Création de contenu multimédia} \hfill (Projets personnels)
%%	\vspace{0.05in}
%%	\begin{itemize} \itemsep -2pt
%%		\item Développement de multiples jeux Android en Java avec la librairie LibGDX;
%%		\item Création d’un clone de Geometry Wars en C\# avec le cadriciel XNA;
%%		\item Publication d’une application Windows Phone en C\#.
%%	\end{itemize}
%%
%%	% Carrefour
%%	Résidences du Carrefour, Saint-Jean-sur-Richelieu
%%	\hfill septembre à mars 2013 \\
%%	{\sl Chef d'équipe} \hfill (Projet de fin d'études collégiales)
%%	\vspace{0.05in}
%%
%%	\begin{itemize} \itemsep -2pt
%%		\item Création d’une application web avec Java et JSP pour un client;
%%		\item Conduire des sessions de travail avec le client pour définir les besoins;
%%		\item Présenter l’offre finale au client et traiter les demandes de changements.
%%	\end{itemize}
%%
%%}
%
%\inEnglish{
%	SORT: Sustainability through Object Recognition and Training
%	\hfill January to April 2017 \\
%	{\sl Technical leader} \hfill (End of university project)
%	\vspace{0.05in}
%	\begin{itemize} \itemsep -2pt
%		\item Creation of an application that help students on sorting waste items.
%		\item Search for object recognition software and techniques (OpenCV, Yolo).
%		\item Deployment of a multi-container architecture with Docker and Docker-compose.
%		\item Management of an open-source project on GitHub.
%	\end{itemize}
%
%	Member of Formule ÉTS (racing car club) \\
%	{\sl developer} \hfill (Student club)
%	\vspace{0.05in}
%	\begin{itemize} \itemsep -2pt
%		\item Creation of a vision and software requirements documents for an application of real time telemetry.
%		\item Prototyping of the application in JavaScript.
%	\end{itemize}
%
%	Member of Conjure (video game creation club) \\
%	{\sl developer} \hfill (Student club)
%	\vspace{0.05in}
%	\begin{itemize} \itemsep -2pt
%		\item Creation of a game in C\# with Unity.
%		\item Game development of multiple Android game in Java.
%		\item Creation of a Geometry Wars clone in C\# with XNA framework.
%		\item Release of a Windows Phone application in C\#.
%		\item Project management.
%	\end{itemize}
%
%	% {\sl Cegep involvement} \hfill (Cegep)
%	% \vspace{0.05in}
%	% \begin{itemize} \itemsep -2pt
%	% 	\item Interview with regional newspaper for the promotion of my course of study.
%	% 	\item Presentation of the projects I have done in school to promote my program of study during the college open day.
%	% \end{itemize}
%
%	% Creation of multimedia content \\
%	% {\sl Content creator} \hfill (Personal projects)
%	% \vspace{0.05in}
%	% \begin{itemize} \itemsep -2pt
%	% 	\item Game development of multiples Android game in Java.
%	% 	\item Creation of a Geometry Wars clone in C\# with XNA framework.
%	% 	\item Release of a Windows Phone application in C\#.
%	% \end{itemize}
%
%	% Carrefour
%	Résidences du Carrefour, Saint-Jean-sur-Richelieu
%	\hfill September to March 2013 \\
%	{\sl Team leader} \hfill (End of Cegep project)
%	\vspace{0.05in}
%
%	\begin{itemize} \itemsep -2pt
%		\item Meeting with clients, elicitation of software requirements.
%		\item Validation of the final offer and handle change requests.
%		\item Creation of a servlet application for the client (Java, JSP).
%	\end{itemize}
%}

% --------------------------------
\inFrench{
	\section{Expériences connexes}
}
\inEnglish{
	\section{Other experiences}
}
\vspace{6pt}

%\inFrench{
%	{\sl Chef d'équipe, secteur restauration}
%		\hfill mai à août 2011, 2012 et 2013 \\
%	Parc d'attractions La Ronde
%
%	\vspace{0.05in}
%	\begin{itemize} \itemsep -2pt
%		\item Gestion et formation continue de 8 à 12 personnes;
%		\item Gestion des inventaires, des commandes et du fonctionnement de quatre points de vente;
%		\item Responsable d’une partie de la communication entre les superviseurs et les préposés;
%		\item Formation des nouveaux employés.
%	\end{itemize}
%
%
%	{\sl Cuisinier, secteur Restauration}
%		\hfill mai à août 2010 \\
%	Parc d'attractions La Ronde
%
%	\vspace{0.05in}
%	\begin{itemize} \itemsep -2pt
%		\item Gestion des périodes d’achalandages et de mon espace de travail.
%	\end{itemize}
%}

\inEnglish{
	{\sl IT Support}
		\hfill 20 Months between 2008 and 2012 \\
	Brain Injury Services of Hamilton, \\
	Agriculture and Agri-food Canada (Guelph, ON), \\
	University of Waterloo School of Pharmacy

	\vspace{0.05in}
	\begin{itemize} \itemsep -2pt
		\item Provided first and second level desktop support in person to variety of different clients including social workers, businessmen, scientists, professors, and students
		\item Produced instructional documentation for common support requests
		\item Discovered and suggested workflow improvements to users to eliminate long tedious tasks
	\end{itemize}


% 	{\sl Cook, restauration sector}
% 		\hfill May to August 2010 \\
% 	Amusement park, La Ronde

% 	\vspace{0.05in}
% 	\begin{itemize} \itemsep -2pt
% 		\item Management of the rush hours and of my work space.
% 	\end{itemize}
}

\vspace{0.1in}

\end{resume}
\end{document}
